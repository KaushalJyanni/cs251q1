\documentclass[]{article}
% \usepackage{datetime}
% \newdate{date}{06}{03}{2017}
\usepackage{graphicx}
\usepackage{hyperref}
\usepackage{cite}
\usepackage{amsmath}
\usepackage{algorithm}
\usepackage{algorithmic	}
\begin{document}

\begin{titlepage}
\center
\newcommand{\HRule}{\rule{\linewidth}{0.5mm}}
\textsc{\LARGE CS251}\\[0.5cm]
\textsc{\Large Assignment 7}\\[0.5cm]
\HRule \\[0.4cm]
{ \huge \bfseries Multiplication Techniques}\\	 % Title of your document
\HRule \\[1.5cm]
\large\emph{Author:}
\textsc{Kaushal Jyanni}\\
% \emph{Date:} \displaydate{date}\\[2cm]
\end{titlepage}

\section{Introduction}
There are various multiplication techniques in vedic multiplication. These techniques do not work on all numbers but in cases that it does, it provides a great way to perform fast multiplication. Some of them are described  below.

\section{Technique 1}~\cite{wiki}

This one works on special cases of two digit numbers; the numbers whose first digits are same and the last digits add to 10. For example 66 and 64, $6=6$, and $6+4=10$. The standard multiplication follows in the following way:\\
\begin{center}\includegraphics[scale=0.5]{standard1.pdf}\end{center}
But in this method we multiply the last two digits and write the product.Multiply the first digit with its successor and write their product. Now concatenate the two products.
\begin{center}\includegraphics[scale=0.5]{firstsame.pdf}\end{center}
If the number of digits in second part is not two then we can add suitable number of zeroes.
For example:\\
\begin{align*}
31 \times 39 &= 3\times(3+1):1\times9\\
&=1209\\
\end{align*}

\subsection{Mathematical Reasoning}
Below is the mathematical explanation of why this method actually works or in other words the proof of correctness.\\
Let $xy$ and $ab$ be the two numbers, where x,y and a,b respectively denotes the two digits of the numbers.

\begin{align}
a = x \label{eq:1}\\
b + y =10 \label{eq:2}
\end{align}

\begin{align*}
xy \times ab&=(10x+y) \times (10a+b) \\
&= 100ax + 10bx + 10ay + by\\
&= 100a^2 + 10a(b+y) + by \tag*{using (\ref{eq:1})}\\
&= 100a^2 + 100a + by \tag*{using (\ref{eq:2})}\\
&= 100a(a + 1) + by
\end{align*}

Hence the above specified technique is correct.



\end{document}
