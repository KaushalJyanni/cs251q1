\documentclass[]{article}
% \usepackage{datetime}
% \newdate{date}{06}{03}{2017}
\usepackage{graphicx}
\usepackage{hyperref}
\usepackage{cite}
\usepackage{amsmath}
\usepackage{algorithm}
\usepackage{algorithmic	}
\begin{document}

\begin{titlepage}
\center
\newcommand{\HRule}{\rule{\linewidth}{0.5mm}}
\textsc{\LARGE CS251}\\[0.5cm]
\textsc{\Large Assignment 7}\\[0.5cm]
\HRule \\[0.4cm]
{ \huge \bfseries Multiplication Techniques}\\	 % Title of your document
\HRule \\[1.5cm]
\large\emph{Author:}
\textsc{Kaushal Jyanni}\\
% \emph{Date:} \displaydate{date}\\[2cm]
\end{titlepage}

\section{Introduction}
There are various multiplication techniques in vedic multiplication. These techniques do not work on all numbers but in cases that it does, it provides a great way to perform fast multiplication. Some of them are described  below.

\section{Technique 1}~\cite{wiki}

This one works on special cases of two digit numbers; the numbers whose first digits are same and the last digits add to 10. For example 66 and 64, $6=6$, and $6+4=10$. The standard multiplication follows in the following way:\\
\begin{center}\includegraphics[scale=0.5]{standard1.pdf}\end{center}
But in this method we multiply the last two digits and write the product.Multiply the first digit with its successor and write their product. Now concatenate the two products.
\begin{center}\includegraphics[scale=0.5]{firstsame.pdf}\end{center}
If the number of digits in second part is not two then we can add suitable number of zeroes.
For example:\\
\begin{align*}
31 \times 39 &= 3\times(3+1):1\times9\\
&=1209\\
\end{align*}

\subsection{Mathematical Reasoning}
Below is the mathematical explanation of why this method actually works or in other words the proof of correctness.\\
Let $xy$ and $ab$ be the two numbers, where x,y and a,b respectively denotes the two digits of the numbers.

\begin{align}
a = x \label{eq:1}\\
b + y =10 \label{eq:2}
\end{align}

\begin{align*}
xy \times ab&=(10x+y) \times (10a+b) \\
&= 100ax + 10bx + 10ay + by\\
&= 100a^2 + 10a(b+y) + by \tag*{using (\ref{eq:1})}\\
&= 100a^2 + 100a + by \tag*{using (\ref{eq:2})}\\
&= 100a(a + 1) + by
\end{align*}

Hence the above specified technique is correct.

\subsection{Pseudocode}
\begin{algorithm}
\caption{Calculate $x \times y$}
\begin{algorithmic} 
\REQUIRE $x_0 = y_0$ and $x_1 + y_1 = 10$
\STATE $x_0 \leftarrow x/10$
\STATE $x_1 \leftarrow x$ mod $10$
\STATE $y_0 \leftarrow y/10$
\STATE $y_1 \leftarrow y$ mod $10$
\STATE $r \leftarrow x_1 \times y_1$
\STATE $l \leftarrow x_0 \times (x_0 + 1)$
\STATE Return $100 \times l + r$
\end{algorithmic}
\end{algorithm}

\subsection{Extension For More Than Two Digits}
Let the two no.s be x and y such that:
\begin{align*}
x &= x_nx_{n-1}.....x_{k+1}x_kx_{k-1}......x_1x_0\\
y &=  y_ny_{n-1}.....y_{k+1}y_ky_{k-1}......y_1y_0
\end{align*}
where $x_i$ and $y_i$ are respectively the $i^{th}$ digit from units place of x and y respectively.\\
IF\\
\begin{align*}
&x_kx_{k-1}......x_1x_0 + y_ky_{k-1}......y_1y_0 = 10^{k+1}\\
&x_nx_{n-1}.....x_{k+1} = y_ny_{n-1}.....y_{k+1}
\end{align*}
then we can use this technique to calculate the multiple.\\
\begin{align*}
a_1 &=  x_nx_{n-1}.....x_{k+1}\\
a_0 &=  x_kx_{k-1}......x_1x_0\\
b_1 &= y_ny_{n-1}.....y_{k+1}\\
b_0 &= y_ky_{k-1}......y_1y_0\\
Therefore:\\
x \times y &= a_1 \times (a_1 + 1) : a_0 \times b_0
\end{align*}

\subsection{Comparison}
\begin{table}[!h]
\begin{center}
\begin{tabular}{|c|c|}
\hline
Standard Method& New Technique\\
\hline
\includegraphics[scale=0.5]{standard1.pdf}&\includegraphics[scale=0.5]{firstsame.pdf}\\
\hline
\end{tabular}
\end{center}
\end{table}

\section{Technique 2}~\cite{Atharvaveda}
This method works on those two digit numbers whose first digits add to 10 and last digits are same. For example 34 and 74, $3 + 7 = 10$ and $4 = 4$. The standard multiplication follows in the following way:\\
\begin{center}\includegraphics[scale=0.5]{standard2.pdf}\end{center}
In this technique we multiply the first digit of both numbers and add the second digit; this forms the first part of the answer. The second part is the square of the second digit. Then concatenate the two parts to get the final answer.
\begin{center}\includegraphics[scale=0.5]{firstsame.pdf}\end{center}

\subsection{Mathematical Reasoning}
Below is the mathematical explanation of why this method actually works or in other words the proof of correctness.\\
Let $xy$ and $ab$ be the two numbers, where x,y and a,b respectively denotes the two digits of the numbers.
\begin{align}
a + x = 10 \label{eq:3}\\
b = y \label{eq:4}
\end{align}

\begin{align*} 
xy \times ab&=(10x+y) \times (10a+b) \\ 
&= 100ax + 10bx + 10ay + by\\
&= 100ax + 10b(a+x) + b^2 \tag*{using (\ref{eq:4})}\\
&= 100ax + 100b + b^2 \tag*{using (\ref{eq:3})}\\
&= 100(ax + b) + b^2
\end{align*}

Hence the above specified technique is correct.
%\newpage



\end{document}
